from openturns import *
from openturns.viewer import ViewImage
from TcDistribution import *

# First, build the elementary duration distributions
d = list()
d.append(IntegralUserDefined([1, 2, 4, 7], [0.1, 0.2, 0.3, 0.4]))
d.append(IntegralUserDefined([2, 3, 5, 9], [0.1, 0.5, 0.2, 0.2]))
d.append(IntegralUserDefined([1, 2, 10], [0.2, 0.2, 0.6]))
# Second, build the lambdas
lambdas = NumericalPoint()
lambdas.add(0.2)
lambdas.add(0.3)
lambdas.add(0.4)
# Third, build t
t = 2.5
# Build the Tc distribution
myTc = TcDistribution(d, lambdas, t)


xMin = -10.0
xMax = 70.0

# Trace de la loi de Tc
gPDF = myTc.drawPDF(xMin, xMax)

drawPDF = gPDF.getDrawable(0)
drawPDF.setLegendName("")
gPDF.setDrawable(drawPDF,0)
gPDF.setTitle("Loi de myTc (calcul analytique)")
gPDF.setXTitle("support de myTc")
gPDF.setYTitle("poids du support")
Show(gPDF)
gPDF.draw("myTc_PDF")

# Trace de la CDF de Tc
gCDF = myTc.drawCDF(xMin, xMax)

drawCDF = gCDF.getDrawable(0)
drawCDF.setLegendName("")
gCDF.setDrawable(drawCDF,0)
gCDF.setTitle("CDF de myTc (calcul analytique)")
gCDF.setXTitle("support de myTc")
gCDF.setYTitle("CDF")
Show(gCDF)
gCDF.draw("myTc_CDF")

print "poids de la valeur ", 5, "=", myTc.computePDF(5)

print "CDF en ", 5, "=", myTc.computeCDF(5)
